\section{Data informativity framework}
% Abstract of the section
In this section we will introduce the notion of informativity well as the notational conventions that will be in force throughout the thesis.

\subsection{Model classes}
We begin with defining model classes. When considering linear systems in state space form we know that they come in different forms. One of the more general forms is the following:
\begin{align}
	\mathbf{x}(t+1) &= A \mathbf{x}(t) + B \mathbf{u}(t) + \mathbf{w}(t)\\
	\mathbf{y}(t+1) &= C \mathbf{x}(t) + D \mathbf{u}(t)
\end{align}

Let us assume we have a system that has no input ($u$), output ($y$) or noise ($w$) variables and has a state space dimension of $3$. Then the system $\Sigma(A,B,C,D)$ can be described using only an $A$ matrix of size $3$. Intuitively this is a very different type of system compared to a system that has a state space that is only $1$ dimension and does contain an input, output and noise variable. Hence we say that they are a different model class. We will note that systems are of the same model class if their state, input and output have the same dimension.
%Hence we define a model class in the following way.
%\Def{Model class}{
%	A system $\Sigma_1$ is said to be the same model class as $\Sigma_2$ if the dimension of the state space, input space and output space are equivalent between both systems.
%}
%If we apply this to the previous example we can see that the 2 systems are not of the same model class because their state spaces have different dimensions.

\subsection{Informativity}
% Introducing the set of systems data informativity
Let $\Sigma$ be the set of discrete-time models of a given model class and let $\mathcal{S}$ be a system from that model class. Then we know that $\mathcal{S}$ is contained in $\Sigma$. suppose that we want to infer a property of the system $\mathcal{S}$. Then if we show that the property holds for all systems in $\Sigma$ then it would also hold for $\mathcal{S}$. However, due to the size of $\Sigma$ this is not very feasible. Thus we need to reduce $\Sigma$ to a more manageable set. To do this we will use the data generated by $\mathcal{S}$. We will call this data $\mathcal{D}$. We will define the set of all systems of the given model class that are able to generate the data $\mathcal{D}$ by $\Sigma_\mathcal{D}$. By construction we know that $\mathcal{S} \in \Sigma_\mathcal{D} \subseteq \Sigma$.

Suppose we want to show if the true system is controllable, we might not be able to uniquely identify the true system using the data, i.e. $\# \Sigma_\mathcal{D} > 1$. However if we are able to show that every system in the set $\Sigma_\mathcal{D}$ is controllable, then we would also know that the true system is controllable. This is the idea behind data informativity. We say the data $\mathcal{D}$ is informative for a property $\mathcal{P}$ if all systems that describe the data $\Sigma_\mathcal{D}$ have the property $\mathcal{P}$. Let $\Sigma_\mathcal{P} \subseteq \Sigma$ be the set of all systems that have the property $\mathcal{P}$. Then we can reformulate the definition of data informativity as follows.

\Def{Informativity \cite[Def 1]{waarde2019data}}{
	We say that the data $\mathcal{D}$ is informative for a property $\mathcal{P}$ if $\Sigma_\mathcal{D} \subseteq \Sigma_\mathcal{P}$.
}

Suppose that the data describes only the true system, i.e. $\Sigma_{\mathcal{D}} = \{ \mathcal{S} \}$ then we know that if a property $\mathcal{P}$ hold for $\mathcal{S}$ then the data is informative for that property. However, in general, just because $\mathcal{S}$ has a property $\mathcal{P}$ does not immediately imply that the data is informative for $\mathcal{P}$ since the data might describe more than one system. Later on in section (\ref{ExampleOfSingleSystemHavingPropertyButNotInformative}) we will see this in an example where the data describes infinity many systems. 

We will also look at control problems. Suppose we want to see if the property '\textit{is stable in full state feedback with a controller $K$}' holds on the data, then we need to know if all systems are stabilisable by state feedback using the controller $K$. For this we will define the set of systems that are stabilised using state feedback for the controller $K$ as follows:
\begin{equation} \label{SigmaK}
	\Sigma_K = \{ (A,B) \, | \, A + B \, K \mbox{ is stable} \}
\end{equation}
Then we have that the data is informative if $\Sigma_\mathcal{D} \subseteq \Sigma_K$. We will generalise this using the following definition.

\Def{Informativity for control \cite[Def 3]{waarde2019data}}{
	We say that the data $\mathcal{D}$ is informative for a property $\mathcal{P}(\cdot)$ if there exists a controller $\mathcal{K}$ such that $\Sigma_\mathcal{D} \subseteq \Sigma_{\mathcal{P}(\mathcal{K})}$.
}


\subsection{Data}
% Introducing the notation for data
We will use the following example \cite[Ex 2]{waarde2019data} to give a more precise definition of data.

\subsubsection*{Input state systems}
Let us consider systems from the model class of state space dimension $n$ and input space dimension $m$ without noise or output. Then we know that all systems contained in $\Sigma$ are of the form:
\begin{equation}
	\label{isSystem}
	\mathbf{x}(t+1) = A \mathbf{x}(t) + B \mathbf{u}(t)
\end{equation}
where $\mathbf{x}(t)$ is the $n$-dimensional state vector and $\mathbf{u}(t)$ is the $m$-dimensional input vector evaluated at time $t$. We will pick a system from $\Sigma$ and call it our 'true' system. We will denote this system by $(A_s , B_s)$. 
We will use our true system to generate/measure the input and state data on $q$ time intervals $\{0,1,\dots,T_i\}$ where $i \in \{1,2,\dots,q\}$. We denote the data collected on one of these intervals as follows:
\begin{align*}
	U^{i}_{-} &= \left[ \begin{array}{cccc} u^{i}(0) & u^{i}(1) & \cdots & u^{i}(T_i - 1) \end{array} \right] \\
	X^{i}     &= \left[ \begin{array}{cccc} x^{i}(0) & x^{i}(1) & \cdots & x^{i}(T_i) \end{array} \right]
\end{align*}
We will now 'split' the state data into a 'past' and 'future' segments, these are defined similar to $U^i_-$:
\begin{align*}
	X^{i}_{-} &= \left[ \begin{array}{ccc} x^{i}(0) & \cdots & x^{i}(T_i - 1) \end{array} \right] \\
	X^{i}_{+} &= \left[ \begin{array}{ccc} x^{i}(1) & \cdots & x^{i}(T_i) \end{array} \right]
\end{align*}
With this representation of our state and input data we have that $X^{i}_{+} = A_s \, X^{i}_{-} + B_s \, U^{i}_{-}$. This holds for all measured intervals $i$ of the true system. We will now combine the data of all intervals to get a more general form:
\begin{align*}
	U_{-} &= \left[ \begin{array}{ccc} U^{1}_{-} & \cdots & U^{q}_{-} \end{array} \right] &
	X     &= \left[ \begin{array}{ccc} X^{1}     & \cdots & X^{q}     \end{array} \right] \\
	X_{+} &= \left[ \begin{array}{ccc} X^{1}_{+} & \cdots & X^{q}_{+} \end{array} \right] &
	X_{-} &= \left[ \begin{array}{ccc} X^{1}_{-} & \cdots & X^{q}_{-} \end{array} \right]
\end{align*}
We will define our data $\mathcal{D} := (U_-, X)$. In this example we have that $\Sigma_\mathcal{D} = \Sigma_{(U_-,X)} = \Sigma_{i/s}$ and is defined by:
\begin{equation} \label{isSet}
	\Sigma_{i/s} = \Sigma_{(U_-,X)} = \left\{ (A, B) \, | \, X_{+} = \begin{bmatrix} A & B \end{bmatrix} \begin{bmatrix} X_{-} \\ U_{-} \end{bmatrix} \right\}
\end{equation}

By construction we know that at least the true system $(A_s,B_s)$ is contained in this set.


\subsubsection*{Input state output systems}
We can extend this concept to also include the output of a system. Assume we have a system of the form:
\begin{subequations}\label{isoSystem}
	\begin{align}
		\mathbf{x}(t+1) &= A \mathbf{x}(t) + B \mathbf{u}(t) \\
		\mathbf{y}(t+1) &= C \mathbf{x}(t) + D \mathbf{u}(t)
	\end{align}
\end{subequations}

Let us define $Y_-$ in the following way:
\begin{align*}
	Y_-^i &= \begin{bmatrix} y^i(0) & y^i(1)& \cdots & y^i(T_i-1); \end{bmatrix}\\
	Y_-   &= \begin{bmatrix} Y_-^1  & Y_-^2 & \cdots & Y_-^q	\end{bmatrix}
\end{align*}
Then we can define the set of systems that can describe the data as follows:
\begin{equation}
\label{isoSet} 
\Sigma_{i/o/s} = 
\Sigma_{(U_-,X, Y_-)} = \left\{ (A, B, C, D) \, | \, 
\begin{bmatrix} X_{+} \\ Y_{-} \end{bmatrix} = 
\begin{bmatrix} A & B \\ C & D \end{bmatrix} 
\begin{bmatrix} X_{-} \\ U_{-} \end{bmatrix} \right\} 
\end{equation}


\subsubsection*{Input state noise systems}
Lastly we will consider systems with bounded noise. These systems will be of the form:
\begin{align} \label{isnSystem}
	\mathbf{x}(t+1) &= A_s \mathbf{x}(t) + B_s \mathbf{u}(t) + \mathbf{w}(t)
\end{align}
%For these systems we will only use a single measurement interval. Hence we will define our noise data $W_-$ as follows. \todo{Verify if it really is one interval}
Let us define the noise $W_-$ n the following way:
\begin{align*}
W_-^i &= \begin{bmatrix} w^i(0) & w^i(1)& \cdots & w^i(T_i-1); \end{bmatrix}\\
W_-   &= \begin{bmatrix} W_-^1  & W_-^2 & \cdots & W_-^q	\end{bmatrix}
\end{align*}
Note that $W_-$ is unknown, we only have access to our state and input data generated by these systems. Later, in section (\ref{sectionNoise}) we will go into more detail on how we define the bound on the noise and how we can use this for control.









