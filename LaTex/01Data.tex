\section{Definitions}
% Abstract of the section
In this section we will introduce the notion of informativity. We will also introduce ourselves with the notation that will be used in the paper to indicate data and systems. 

\subsection{Informativity}
% Introducing the set of systems data informativity
Let $\Sigma$ be the set of discrete time models with a given state space dimension $n$ and a given input space dimension $m$. Then if we have state/input data $\mathcal{D}$ of this form, then we know that the system $\mathcal{S}$ that produced this data is contained in the set $\Sigma$. However, using $\Sigma$ is not very useful unless we reduce the set to be more manageable, we can do this by using our knowledge of the true system, namely the data. We will only consider systems that are able to produce the data, we will call this set $\Sigma_\mathcal{D} \subseteq \Sigma$. 

Suppose we want to show if the true system is controllable, we might not be able to uniquely identify the true system using the data, i.e. $\# \Sigma_\mathcal{D} > 1$. However if we are able to show that every system in the set $\Sigma_\mathcal{D}$ is controllable, then we would also know that the true system is controllable. This is the idea behind data informativity. We say the data $\mathcal{D}$ is informative for a property $\mathcal{P}$ if all systems that describe the data $\Sigma_\mathcal{D}$ have the property $\mathcal{P}$. Let $\Sigma_\mathcal{P} \subseteq \Sigma$ be the set of all systems that have the property $\mathcal{P}$. Then we can reformulate the definition of data informativity as follows.

\Def{Informativity \cite[Def 1]{waarde2019data}}{
	We say that the data $\mathcal{D}$ is informative for a property $\mathcal{P}$ if $\Sigma_\mathcal{D} \subseteq \Sigma_\mathcal{P}$.
}

Suppose that the data describes only the true system, i.e. $\Sigma_{(U_-,X)} = \{ \mathcal{S} \}$ then we know that if a property $\mathcal{P}$ hold for $\mathcal{S}$ then the data is informative for that property. However, in general, just because $\mathcal{S}$ has a property $\mathcal{P}$ does not immediately imply that the data is informative for $\mathcal{P}$ since the data might describe more then one system. Later on we will see this in an example where the data describes infinity many systems. \todo{add reference to example with no identification and state feedback}

We will also focus on control problems. Suppose we want to see if the property 'is stable in full state feedback with a controller $K$' holds on the data, then we need to know if all systems are stabilisable by state feedback using the controller $K$. For this we will define the set of systems that are stabilised using state feedback for the controller $K$ as follows:
\[ \Sigma_K = \{ (A,B) \, | \, A + B \, K \mbox{ is stable} \} \]
Then we have that the data is informative if $\Sigma_\mathcal{D} \subseteq \Sigma_K$. We will generalise this using the following definition.

\Def{Informativity for control \cite[Def 3]{waarde2019data}}{
	We say that the data $\mathcal{D}$ is informative for a property $\mathcal{P}(\cdot)$ if there exists a controller $\mathcal{K}$ such that $\Sigma_\mathcal{D} \subseteq \Sigma_{\mathcal{P}(\mathcal{K})}$.
}


\subsection{Data}
% Introducing the notation for data
We will use the following example \cite[Ex 2]{waarde2019data} to give a more precise definition of data.


Let $n$ be the dimension of the state space and $m$ be the dimension of the input space. Assume that both are known. Then we know that all systems contained in $\Sigma$ are of the form:
\begin{equation}
	\label{isSystem}
	\mathbf{x}(t+1) = A \mathbf{x}(t) + B \mathbf{u}(t)
\end{equation}
Where $\mathbf{x}(t)$ is the $n$-dimensional state vector and $\mathbf{u}(t)$ is the $m$-dimensional input vector evaluated at time $t$. We will assume our data is generated by the 'true' system, we will denote this system as $(A_s , B_s)$. We will now measure the input and state data on $q$ time intervals $\{0,1,\dots,T_i\}$ where $i \in \{1,2,\dots,q\}$. We denote the data collected on one of these intervals as follows:
\begin{align*}
	U^{i}_{-} &= \left[ \begin{array}{cccc} u^{i}(0) & u^{i}(1) & \dots & u^{i}(T_i - 1) \end{array} \right] \\
	X^{i}     &= \left[ \begin{array}{cccc} x^{i}(0) & x^{i}(1) & \dots & x^{i}(T_i) \end{array} \right]
\end{align*}
We will now 'split' the state data into a 'past' and 'future' segment, these are defined similar to $U^i_-$.
\begin{align*}
	X^{i}_{-} &= \left[ \begin{array}{ccc} x^{i}(0) & \dots & x^{i}(T_i - 1) \end{array} \right] \\
	X^{i}_{+} &= \left[ \begin{array}{ccc} x^{i}(1) & \dots & x^{i}(T_i) \end{array} \right]
\end{align*}
With this representation of our state and input data we have that $X^{i}_{+} = A_s \, X^{i}_{-} + B_s \, U^{i}_{-}$. This holds for all measured intervals $i$ of the true system. We will now combine the data of all intervals to get a more general form.
\begin{align*}
	U_{-} &= \left[ \begin{array}{ccc} U^{1}_{-} & \dots & U^{q}_{-} \end{array} \right] &
	X     &= \left[ \begin{array}{ccc} X^{1} & \dots & X^{q} \end{array} \right] \\
	X_{+} &= \left[ \begin{array}{ccc} X^{1}_{+} & \dots & X^{q}_{+} \end{array} \right] &
	X_{-} &= \left[ \begin{array}{ccc} X^{1}_{-} & \dots & X^{q}_{-} \end{array} \right]
\end{align*}
We will define our data $\mathcal{D} := (U_-, X)$. In this example we have that $\Sigma_\mathcal{D} = \Sigma_{(U_-,X)} = \Sigma_{i/s}$ and is defined by:
\[ \Sigma_{(U_-,X)} = \left\{ (A, B) \, | \, X_{+} = \begin{bmatrix} A & B \end{bmatrix} \begin{bmatrix} X_{-} \\ U_{-} \end{bmatrix} \right\} \]
By construction we know that at least the true system $(A_s,B_s)$ is contained in this set.

We can extend this concept to also include the output of a system. Assume we have a system of the form:
\begin{align}
	\label{isoSystem}
	\mathbf{x}(t+1) &= A \mathbf{x}(t) + B \mathbf{u}(t) \\
	\mathbf{y}(t+1) &= C \mathbf{x}(t) + D \mathbf{u}(t)
\end{align}
Let us define $Y_-$ in the following way:
\begin{align*}
	Y_-^i &= \begin{bmatrix}	y^i(0) & y^i(1)& \dots & y^i(T_i-1); \end{bmatrix}\\
	Y_- &= \begin{bmatrix} Y_-^1 & Y_-^2 & \dots & Y_-^q	\end{bmatrix}
\end{align*}
Then we can define the set of systems that can describe the data as follows:
\begin{equation}
\label{isoSet} 
\Sigma_{(U_-,X, Y_-)} = \left\{ (A, B, C, D) \, | \, 
\begin{bmatrix} X_{+} \\ Y_{-} \end{bmatrix} = 
\begin{bmatrix} A & B \\ C & D \end{bmatrix} 
\begin{bmatrix} X_{-} \\ U_{-} \end{bmatrix} \right\} 
\end{equation}

% Combining both consepts for a more precise defenition
