\section{Controllability}
% Abstract section
In this section we will see how we can use the data to conclude if a set of systems are controllable of not. We will first consider the mathematics and then we will see how this can be implemented. Lastly we will consider an example that we will solve using the Matlab function based on the previously discussed algorithm.

% What is controllability
\Def{Informative for controllability \cite[Def 7]{waarde2019data}}{
	We say that the data is informative for controllability if all systems that describe the data are controllable. I.e. $\Sigma_{i/o} \subseteq \left\{ (A,B) \, | \, (A,B) \mbox{ is controllable} \right\}$.
}

% Mathematics
We will base our algorithm on theorem 8 and remark 9 from \cite{waarde2019data}. These give necessary and sufficient conditions for the informativity.

\Thr{Informative for controllability}{
	The data $(U_-, X)$ is informative for controllability if and only if $rank(X_+ - \lambda X_- ) = n$ $\forall \lambda \in \mathbb{C}$.
}

This theorem can be reduced to check a finite amount of complex numbers similar to the Hautus test. We have that the above theorem is equivalent to $rank(X+) = n$ and $rank(X_+ - \lambda X_-) = n$ for all $\lambda \neq 0$ with $\lambda^{-1} \in \sigma(X_- X_+^\dagger)$ with $X_+^\dagger$ being the right inverse of $X_+$ and $\sigma(\cdot)$ denotes the set of eigenvalues of the matrix.

% Proof?
%\todo{proof?}

% Pseudo code / algorithm
%This will result in the following pseudo code for the algorithm:
%\begin{lstlisting}
%provide X_ and X+
%if X_ has full row rank
%	for each non zero eigenvalue (lambda) of X_ * X+^-i (X+^-i : right inverse of X+) 
%	check if rank(X+ - \lambda X_) = n
%	if it holds for all eigenvalues then
%		The data is informative for controllability
%else 
%	The data is not informative for controllability
%\end{lstlisting}

% Example using function
\subsection{Implementation}
The algorithm above is implemented in the following functions:
\subsubsection*{Syntax}
\mon{[bool] = isInformControllable(X)} 

\subsubsection*{Description}
\mon{[bool] = isInformControllable(X)}: Returns if the state data from an input-state dataset is informative for controllability.

\subsubsection*{Input arguments}
\textbf{\mon{X}}: State data matrix of dimension $n \times T+1$ from an input-state dataset.

\subsubsection*{Output arguments}
\textbf{\mon{bool}}: (boolean) True if the data is informative for controllability, false otherwise

\subsection{Examples} \label{NotIdentifyButControllable}
We will consider the same data as we did in the example of system identification. Recall the provided data was:
\begin{align*}
X &= \begin{bmatrix} 0&1&0 \\ 0&0&1 \end{bmatrix} & U = \begin{bmatrix}	1&0	\end{bmatrix}
\end{align*} 
For the data to be informative for controllability we need that $X_+ - \lambda X_-$ is full row rank for all $\lambda \in \mathbb{C}$.
\begin{align*}
rank(X_+ - \lambda X_-) = rank\left(\begin{bmatrix} 1&0\\0&1\end{bmatrix} - \begin{bmatrix} 0&\lambda\\0&0\end{bmatrix}\right) = 2
\end{align*}
Hence the data is informative for controllability. 

We can also find the same result by using the Matlab function:
\begin{lstlisting}
X = [0 1 0 ; 0 0 1]; U = [1 0];
[bool] = isInformControllable(X)
\end{lstlisting}
Which will return: \mon{[ 1 ]}.

We can verify the result by considering the systems that generate this data. Recall that the data is generated by systems of the form:
\[ \Sigma_{i/s} = \left\{ \left( \begin{bmatrix} 0&a_1 \\ 1&a_2 \end{bmatrix}, \begin{bmatrix} 1 \\ 0 \end{bmatrix} \right) \, | \, a_1, a_2 \in \R \right\} \]
Thus the controllability matrix is given by:
\[ \begin{bmatrix} B& AB \end{bmatrix} = \begin{bmatrix} 1 & 0 \\ 0 & 1 \end{bmatrix} \]
Since the controllability matrix has full rank we can conclude that systems of this form are indeed controllable.
































