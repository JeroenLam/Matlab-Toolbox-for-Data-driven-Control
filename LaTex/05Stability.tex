\section{Stability}
% Abstract section
In this section we will consider data informativity for stability of unforced systems. We will first see how we can define this notion after which we will show how we can reduce it to a program friendly condition.

% What is Stability
In this section we will consider our data to be obtained from an unforced system i.e. there was no input. Because of this we will define our data and set of systems as follows:
\begin{align*}
\mathcal{D} &= (X) & \Sigma_\mathcal{D} = \left\{ A | X_+ = A \, X_- \right\}
\end{align*}
Using these definitions we can define data informativity for stability as follows.

\Def{Informative for stability}{
	We say the data is informative for stability if all unforced systems describing the data are stable.
}


% Mathematics
\subsection{Mathematics stability}
However, using our definition of informativity for stabilisability we are able to reduce the condition to showing that the data is informative for identification and that the identified system is stable.

\Cor{\cite{Cor 11}{waarde2019data}}{
	The data $X$ is informative for stability if and only if $X_-$ has full row rank and $X_+ X_-^\dagger$ is stable for any right inverse $X_-^\dagger$. This is equivalent to $\Sigma_\mathcal{D} = \left\{A_s\right\}$ and $A_s = X_+ X_-^\dagger$ being stable.
}

% Proof?
Proof '$\Rightarrow$': \\
%Assume that the data is informative for stability. Then we know that all systems describing the data are stable. We defined the set of systems as follows:
%\begin{align*}
%\Sigma_\mathcal{D} &= \left\{ A | X_+ = A \, X_- \mbox{ s.t. $A$ is stable} \right\} & & \\
%\Sigma_\mathcal{D} &= \left\{ A | X_+ \, X_-^\dagger = A \mbox{ s.t. $A$ is stable} \right\} & \iff& X_- \mbox{ is full row rank} 
%\end{align*}
%Hence $A \in \Sigma_\mathcal{D}$

Proof '$\Leftarrow$': \\

% Pseudo code / algorithm
%\subsection{Algorithm stability}
%Using the corollary above we can construct the following algorithm for finding if the state data generated from an unforced system is informative for stability.
%\begin{lstlisting}
%provide X_- and X_+
%if the data is informative for system identification
%    identify the system matrix A
%    if A is stable
%    	The data is informative for stability
%else
%	The data is not informative for stability
%\end{lstlisting}

% Example using function
\subsection{Examples using implementation}
The algorithm above is implemented in the following functions:
\subsubsection*{Syntax}
\mon{[bool] = isInformStable(X)}

\subsubsection*{Description}
\mon{[bool] = isInformStable(X)}: Returns if the data of a unforced system is informative for stability.

\subsubsection*{Input arguments}
\textbf{\mon{X}}: State data matrix of dimension $n \times T+1$.

\subsubsection*{Output arguments}
\textbf{\mon{bool}}: (boolean) True if the data is informative for stability, false otherwise

\subsubsection{Examples}
For this example we will consider the following state data generated by a unforced system:
\[ X = \begin{bmatrix}
1 & \frac{1}{2} & \frac{1}{4} \\ 0 & \frac{1}{2} & \frac{1}{2}
\end{bmatrix} \]
For the data to be informative for stability we need that the data is informative for system identification and that the identified system is stable. The data is informative for system identification if and only if there exists an right inverse $X_-^\dagger$ of $X_-$. Then the identified system is given by $A = X_+ X_-^\dagger$.
\[ A = \begin{bmatrix}
\frac{1}{2} & \frac{1}{4} \\ \frac{1}{2} & \frac{1}{2}
\end{bmatrix} \begin{bmatrix}
1 & \frac{1}{2} \\ 0 & \frac{1}{2}
\end{bmatrix}^{-1} = \begin{bmatrix} \frac{1}{2} & 0 \\ \frac{1}{2} & \frac{1}{2} \end{bmatrix} \]
As we can see from the $A$ matrix its eigenvalues are $\sigma(A) = \{\frac{1}{2}, \frac{1}{2}\}$. Thus the system is stable and hence the data is informative for stability.

We can also find the same result by using the Matlab function:
\begin{lstlisting}
X = [1 0.5 0.25; 0 0.5 0.5];
[bool, A] = isInformIdentification(X)
\end{lstlisting}
Which will return: \mon{[ 1, [0.5 0 ; 0.5 0.5] ]}.