\section{Stabilisability}
% Abstract section
In this section we will see how we can use the data to conclude if a set of systems are stabilisable of not. We will first consider the mathematics and then we will see how this can be implemented. Lastly we will consider an example that we will solve using the Matlab function based on the previously discussed algorithm.

% What is Stabilisability
\Def{Informative for stabilisability \cite[Def 7]{waarde2019data}}{
	We say that the data is informative for stabilisability if all systems that describe the data are stabilisable. I.e. $\Sigma_{i/o} \subseteq \left\{ (A,B) \, | \, (A,B) \mbox{ is stabilisable} \right\}$.
}

% Mathematics
We will base our algorithm on theorem 8 and remark 9 from \cite{waarde2019data}. These give necessary and sufficient conditions for the informativity.

\Thr{Informative for stabilisability}{
	The data $(U_-, X)$ is informative for stabilisability if and only if $rank(X_+ - \lambda X_- ) = n$ $\forall \lambda \in \mathbb{C}$ such that $|\lambda| \geq 1$.
}

This theorem can be reduced to check a finite amount of complex numbers similar to the Hautus test. We have that the above theorem is equivalent to $rank(X_+ - X_-) = n$ and $rank(X_+ - \lambda X_-) = n$ for all $\lambda \neq 1$ with $(\lambda - 1)^{-1} \in \sigma(X_- (X_+ - X_-)^\dagger)$ with $(X_+ - X_-)^\dagger$ being the right inverse of $(X_+ - X_-)$ and $\sigma(\cdot)$ denotes the set of eigenvalues of the matrix.

% Proof?
\todo{proof?}

% Pseudo code / algorithm


% Example using function
\subsection{Examples using implementation}
The algorithm above is implemented in the following functions:
\subsubsection*{Syntax}
\mon{[bool] = isInformStabilisable(X)} \\
\mon{[bool] = isInformStabilisable(X, tolerance)}

\subsubsection*{Description}
\mon{[bool] = isInformStabilisable(X)}: Returns if the data is informative for stabilisability. Uses default tolerance of \mon{1e-14}.\\
\mon{[bool] = isInformStabilisable(X, tolerance)}: Returns if the data is informative for stabilisability given a specific tolerance.  

\subsubsection*{Input arguments}
\textbf{\mon{X}}: State data matrix of dimension $n \times T+1$.\\
\textbf{\mon{tolerance}}: Tolerance used for determining when a value is zero up to machine precision. Default value is \mon{1e-14}.

\subsubsection*{Output arguments}
\textbf{\mon{bool}}: (boolean) True if the data is informative for stabilisability, false otherwise\\

\subsubsection{Examples}
In this example we will consider the same data that we used in system identification and controllability, namely:
\begin{align*}
X &= \begin{bmatrix} 0&1&0 \\ 0&0&1 \end{bmatrix} & U = \begin{bmatrix}	1&0	\end{bmatrix}
\end{align*} 
For the data to be informative for stabilisability we need that $X_+ - \lambda X_-$ is full row rank for all $\lambda \in \mathbb{C}$ such that $|\lambda| \geq 1$.
\begin{align*}
rank(X_+ - \lambda X_-) = rank\left(\begin{bmatrix} 1&0\\0&1\end{bmatrix} - \begin{bmatrix} 0&\lambda\\0&0\end{bmatrix}\right) = 2
\end{align*}
Since the rank condition hold for all $\lambda$ it will also hold for all $|\lambda| \geq 1$. Hence the data is informative for stabilisability. This is consistent with the non data driven theory that controllability implies stabilisability. 

We can also find the same result by using the Matlab function:
\begin{lstlisting}
X = [0 1 0 ; 0 0 1]; U = [1 0];
[bool] = isInformStabilisable(X)
\end{lstlisting}
Which will return: \mon{[ 1 ]}.

We can verify the result by considering the systems that generate this data. Recall that the data is generated by systems of the form:
\[ \Sigma_{i/s} = \left\{ \left( \begin{bmatrix} 0&a_1 \\ 1&a_2 \end{bmatrix}, \begin{bmatrix} 1 \\ 0 \end{bmatrix} \right) \, | \, a_1, a_2 \in \R \right\} \]
Recall that a pair $(A,B)$ is stabilisable if and only if $\begin{bmatrix} A - \lambda I & B \end{bmatrix}$ is full row rank for all $\lambda \in \sigma(A)$ such that $Re(\lambda) \geq 0$. 
\[ rank \left( \begin{bmatrix} A - \lambda I & B \end{bmatrix} \right) = rank \left( \begin{bmatrix} -\lambda & a_1 & 1 \\ 1 & a_2 - \lambda & 0 \end{bmatrix} \right) = 2 \]
Thus the systems that describe the data are stabilisable.