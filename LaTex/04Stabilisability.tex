\section{Stabilisability}
% Abstract section


% What is Stabilisability


% Mathematics


% Proof?


% Pseudo code / algorithm


% Example using function
\subsection{Examples using implementation}
The algorithm above is implemented in the following functions:
\subsubsection*{Syntax}
\mon{[bool] = isInformStabilisable(X)} \\
\mon{[bool] = isInformStabilisable(X, tolerance)}

\subsubsection*{Description}
\mon{[bool] = isInformStabilisable(X)}: Returns if the data is informative for stabilisability. Uses default tolerance of \mon{1e-14}.\\
\mon{[bool] = isInformStabilisable(X, tolerance)}: Returns if the data is informative for stabilisability given a specific tolerance.  

\subsubsection*{Input arguments}
\textbf{\mon{X}}: State data matrix of dimension $n \times T+1$.\\
\textbf{\mon{tolerance}}: Tolerance used for determining when a value is zero up to machine precision. Default value is \mon{1e-14}.

\subsubsection*{Output arguments}
\textbf{\mon{bool}}: (boolean) True if the data is informative for stabilisability, false otherwise\\

\subsubsection{Examples}