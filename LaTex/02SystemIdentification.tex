\section{System identification}
% Abstract section
In this section we will see how we can use the data to identify the true system. We will first consider the theory and then we will see how this can be implemented. Lastly we will consider an example that we will solve using the Matlab function based on the previously discussed algorithm.

% What is Systems identification
\Def{Informative for system identification \cite[Def 5]{waarde2019data}}{
	We say that the data is informative for system identification if the data only describes a single system, i.e. $\Sigma_\mathcal{D} = \{(A_s,B_s)\}$.
}

% Mathematics
\subsection{Theory}
Consider systems of the form (\ref{isSystem}) and let $(U_-,X)$ be the data generated by the true system. Assume that we have recorded $T$ data points, i.e. the dimension of $U_-$ is $m \times T$ and the dimension of $X$ is $n \times (T+1)$. Recall that the set of systems described by this data is given by:
\begin{equation}
	\tag{\ref{isSet}} 
	\Sigma_{(U_-,X)} = \left\{ (A, B) \, | \, X_{+} = \begin{bmatrix} A & B \end{bmatrix} \begin{bmatrix} X_{-} \\ U_{-} \end{bmatrix} \right\} 
\end{equation}
where $\begin{bmatrix} X_{-} \\ U_{-} \end{bmatrix}$ is $(n+m) \times T$. Suppose that the $rank \begin{bmatrix} X_{-} \\ U_{-} \end{bmatrix} = n+m$, then there exists a right inverse $\begin{bmatrix} V_1 & V_2 \end{bmatrix}$ such that $\begin{bmatrix} X_{-} \\ U_{-} \end{bmatrix} \begin{bmatrix} V_1 & V_2 \end{bmatrix} = I_{n+m}$. If we multiply the equation from (\ref{isSet}) with this inverse we get the following:
\begin{equation*}
	X_{+} \begin{bmatrix} V_1 & V_2 \end{bmatrix} = \begin{bmatrix} A & B \end{bmatrix} I_{n+m}
\end{equation*}
Thus if the data is full rank we are able to retrieve the $(A,B)$ pair directly. This results in the following proposition.

% Pseudo code / algorithm
\subsection{Algorithm}
\Prop{\cite[Prop 6]{waarde2019data}}{
	The data $(U_-,X)$ is informative for system identification if and only if:
	\begin{equation} \label{SysIdentRankCond}
		rank \begin{bmatrix} X_{-} \\ U_{-} \end{bmatrix} = n+m 
	\end{equation}
	Furthermore, if (\ref{SysIdentRankCond}) holds, then there exists a right inverse $\begin{bmatrix} V_1 & V_2 \end{bmatrix}$ (as defined above), and for any such right inverse $A_s = X_+ V_1$ and $B_s = X_+ V_2$.
}

Suppose we consider a system without any inputs, then the proposition reduces to checking if $X_-$ has full row rank and finding a right inverse $X_-^\dagger$ of $X_-$. Then we retrieve the system as follows $A_s = X_+ X_-^\dagger$.

let us assume we consider an input, state, output system of the form (\ref{isoSystem}). Recall the set of systems described by the data is given by (\ref{isoSet}):
\begin{equation*} \tag{\ref{isoSet}}
\Sigma_{(U_-,X, Y_-)} = \left\{ (A, B, C, D) \, | \, 
\begin{bmatrix} X_{+} \\ Y_{-} \end{bmatrix} = 
\begin{bmatrix} A & B \\ C & D \end{bmatrix} 
\begin{bmatrix} X_{-} \\ U_{-} \end{bmatrix} \right\} 
\end{equation*}
In this case if the proposition holds, we can also retrieve the $C$ and $D$ matrix by computing $C = Y_- V_1$ and $D = Y_- V_2$.

% Example using function
\subsection{Implementation}
\subsubsection*{Syntax}
\mon{[bool, A] = isInformIdentification(X)} \\
\mon{[bool, A, B] = isInformIdentification(X, U)} \\
\mon{[bool, A, B, C, D] = isInformIdentification(X, U, Y)}

\subsubsection*{Description}
\mon{[bool, A] = isInformIdentification(X)}: returns if the state data is informative for system identification, if it is then \mon{A} contains the \mon{A} is the system matrix in state space representation.\\
\mon{[bool, A, B] = isInformIdentification(X, U)}: returns if the state and input data is informative for system identification, if it is then \mon{A} and \mon{B} are the system matrices in state space representation.\\
\mon{[bool, A, B, C, D] = isInformIdentification(X, U, Y)}: returns if the state, input and output data is informative for system identification, if it is then \mon{A}, \mon{B}, \mon{C} and \mon{D} are the system matrices in state space representation.

\subsubsection*{Input arguments}
\textbf{\mon{X}}: State data matrix of dimension $n \times (T+1)$.\\
\textbf{\mon{U}}: Input data matrix of dimension $m \times T$.\\
\textbf{\mon{Y}}: Output data matrix of dimension $p \times T$.

\subsubsection*{Output arguments}
\textbf{\mon{bool}}: (boolean) True if the data is informative for system identification, false otherwise\\
\textbf{\mon{A}}: (matrix) If the data is informative, it contains the systems \mon{A} matrix, empty otherwise.\\
\textbf{\mon{B}}: (matrix) If the data is informative, it contains the systems \mon{B} matrix, empty otherwise.\\
\textbf{\mon{C}}: (matrix) If the data is informative, it contains the systems \mon{C} matrix, empty otherwise.\\
\textbf{\mon{D}}: (matrix) If the data is informative, it contains the systems \mon{D} matrix, empty otherwise.


\subsection{Examples}
Let us consider the following state and input data:
\begin{align*}
	X &= \begin{bmatrix} 0&1&0 \\ 0&0&1 \end{bmatrix} & U = \begin{bmatrix}	1&0	\end{bmatrix}
\end{align*} 
As we can see the data is not sufficient for system identification:
\begin{equation*}
	rank \begin{bmatrix} X_{-} \\ U_{-} \end{bmatrix}  = rank \begin{bmatrix} 0&1 \\ 0&0 \\ 1&0 \end{bmatrix} = 2 \neq 3
\end{equation*}
This is because the data can be generated by systems of the following form:
\[ \Sigma_{i/s} = \left\{ \left( \begin{bmatrix} 0&a_1 \\ 1&a_2 \end{bmatrix}, \begin{bmatrix} 1 \\ 0 \end{bmatrix} \right) \, | \, a_1, a_2 \in \R \right\} \]
However, if we where to consider the same data with 1 additional data point then the data would be informative for system identification:
\begin{align*}
	X_- &= \begin{bmatrix} 0&1&0 \\ 0&0&1 \end{bmatrix} & U &= \begin{bmatrix}	1&0&\alpha	\end{bmatrix} & rank \begin{bmatrix} 0&1&0 \\ 0&0&1 \\ 1&0&\alpha \end{bmatrix} = 3
\end{align*} 
We can also find the same results using the MatLab functions:
\begin{lstlisting}
X = [0 1 0 ; 0 0 1]; U = [1 0];
[bool, A, B] = isInformIdentification(X, U)
\end{lstlisting}
which will return: \mon{[ false, [], [] ]}.




