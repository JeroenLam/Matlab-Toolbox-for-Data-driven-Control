\section{Noise and control} \label{sectionNoise}
% Abstract section
In this section we will discus how we can define our data in the case where there is an unknown noise term acting on the system. We will formulate an assumption on our noise that we will use to represent the control problem as one of quadratic matrix inequalities. 

\subsection{Noise and its set of systems}
% Recall noise from start
Recall that we defined systems with bounded noise in the following way.
\begin{align} \tag{\ref{isnSystem}}
\mathbf{x}(t+1) &= A_s \mathbf{x}(t) + B_s \mathbf{u}(t) + \mathbf{w}(t)
\end{align}
Where $W_-$ was defined in the same way as $U_-$ and $Y_-$, i.e. it is a matrix where each column represents a sample from the given variable. Using this we can rewrite (\ref{isnSystem}) using the data matrices in the following way.
\begin{equation} \label{isnSystemData}
	X_+ = A_s X_- + B_s U_- + W_-
\end{equation}
Before we consider the set of systems with noise, we will first look at how we can reformulate our boundedness assumption on the noise as a quadratic matrix inequality.

% How do we define noise
\cite[Assumption 1]{waarde2020noisy}
Let the noise samples $w(0),W(1),\dots,w(T-1)$ be collected in the matrix $W_-$, satisfy the bound
\begin{equation} \label{noiseBound}
	\begin{bmatrix} I \\ W_-^\top \end{bmatrix} ^\top
	\begin{bmatrix} \Phi_{11} & \Phi_{12} \\ \Phi_{12}^\top & \Phi_{22} \end{bmatrix}
	\begin{bmatrix} I \\ W_-^\top \end{bmatrix} \geq 0
\end{equation}
for known matrices $\Phi_{11} = \Phi_{11}^\top$, $\Phi_{12}$ and $\Phi_{22} = \Phi_{22}^\top < 0$.

% Note special case of noise matrix
This assumption gives us a \todo{bound on noise, extend}.
We can also look at the special case in which $\Phi_{12} = 0$ and $\Phi_{22} = -I$. In this case we are able to rewrite (\ref{noiseBound}) to get the following.
\begin{equation} \label{noiseBoundSpecialCase}
	W_- W_-^\top = \sum^{T-1}_{t = 0}w(t)w(t)^\top \leq \Phi_{11}
\end{equation}
In this case we can see that the noise needs to have a finite bound to conform to our assumption. 

% Define set of systems
Now that we have introduced our notation, we are able to define the set of systems that are described by some input and state data given that there is noise.
\begin{equation} \label{isnSet}
\Sigma_{i/s/n} = \left\{ (A,B) \, | \, \mbox{(\ref{isnSystemData}) holds for some }W_- \mbox{ satisfying (\ref{noiseBound})} \right\}
\end{equation}

% Write system set as matrix inequality.
We will now substitute (\ref{isnSystemData}) in (\ref{noiseBound}) to get a single quadratic matrix inequality that can be used to define $\Sigma$.
\begin{equation} \label{noiseSystemQMI}
	\begin{bmatrix} I \\ A^\top \\ B^\top \end{bmatrix}^\top
	\begin{bmatrix} I&X_+ \\ 0 & -X_- \\ 0&-U_- \end{bmatrix}
	\begin{bmatrix} \Phi_{11} & \Phi_{12} \\ \Phi_{12}^\top & \Phi_{22} \end{bmatrix}
	\begin{bmatrix} I&X_+ \\ 0 & -X_- \\ 0&-U_- \end{bmatrix}^\top
	\begin{bmatrix} I \\ A^\top \\ B^\top \end{bmatrix} \geq 0
\end{equation}

Using this inequality we can redefine $\Sigma$ as follows.

\Lemma{\cite[Lemma 4]{waarde2020noisy}}{
	We have that $\Sigma = \left\{ (A,B) \, | \, \mbox{(\ref{noiseSystemQMI}) is satisfied} \right\}$.
}

















