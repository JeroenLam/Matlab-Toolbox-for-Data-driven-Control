\section{$\mathcal{H}_2$ Control}
% Abstract section
In this section we will look at the $\mathcal{H}_2$ control control problem for non data-driven control. We will extend this to define informativity and look at a theorem that gives us a methods to compute a solution to the data driven $\mathcal{H}_2$ control problem. Lastly we will consider the implementation as well as an example.

% What is H2 Classical
\subsection{Non data driven $\mathcal{H}_2$ control}
Similar to quadratic stabilisation, we will be considering systems of the following form.
\begin{align} \tag{\ref{isnSystem}}
\mathbf{x}(t+1) &= A_s \mathbf{x}(t) + B_s \mathbf{u}(t) + \mathbf{w}(t)
\end{align}
But in the case of $\mathcal{H}_2$ control we will also be considering an performance output on the system.
\begin{align*}
	\mathbf{z}(t) &= C \mathbf{x}(t) + D \mathbf{u}(t)
\end{align*}
In the case of $\mathcal{H}_2$ control we do not only want to find a stabilising feedback controller, we want to find one that minimises the performance output. In this sense it is similar to LQR control. Let $K$ be a stabilising feedback controller for the system. We will start by defining the closed loop transfer function ($G(z)$) on the closed loop system.
\begin{align*}
	\mathbf{x}(t+1) &= (A + B K) \mathbf{x}(t) + \mathbf{w}(t) \\
	\mathbf{z}(t)   &= (C   + D   K) \mathbf{x}(t) \\
	G(z) &= (C + DK)(z I - (A+BK))^{-1}
\end{align*}
We define the performance of a given controller as $\gamma$. Using this notation we can define the classical $\mathcal{H}_2$ control problem.

\cite[Problem 11.1]{bookTrentelman} Consider systems in the following form as well as the cost function $\left\| G_K(z) \right\|^2_{\mathcal{H}_2}$.
\begin{align*}
	\mathbf{x}(t+1) &= (A_s + B_s K) \mathbf{x}(t) + \mathbf{w}(t) \\
	\mathbf{z}(t)   &= (C   + D   K) \mathbf{x}(t) \\
	G_K(z) &= (C + DK)(z I - (A+BK))^{-1}
\end{align*}
We want to find the feedback gain $K$ for which the cost function is minimal, i.e. let the minimal cost be as follows.
\begin{equation*}
	J^\ast = inf \{ J(K) := \left\| G_K(z) \right\|^2_{\mathcal{H}_2} \, | \, K \mbox{ stabilises the closed loop system}\}
\end{equation*}
Then we want to find $K$ such that $J(K) = J^\ast$. However, instead of using this definition directly, we will be considering the following theorem that gives us an equivalence between the definition and a set of matrix inequalities.

\todo{find reference for statement that is 'well-known' according to paper page 8}
\Thr{}{
	A+BK is stable and $\left\| G_K(z) \right\|^2_{\mathcal{H}_2} < \gamma$ if and only if there exists a symmetric positive definite matrix $P$ such that
	\begin{align} \label{H2Inequalities}
		P &> (A + BK)^\top P (A + BK) + (C+DK)^\top (C+DK) & trace(P) &< \gamma^2
	\end{align}
}



% What is H2 Data-driven
\subsection{Data driven $\mathcal{H}_2$ control}
We will now extend the $\mathcal{H}_2$ control problem to data driven control. We want to find a controller $K$ that stabilises all systems describing the data whilst still optimizing $\gamma$ to be as small as possible. Using the theorem noted above we are able to do so since we can express $\gamma$ in relation to $P$ instead of it being directly dependent on $(A,B) \in \Sigma_{i/s/n}$. Using this we will formulate the following definition of informativity.

\Def{Informative for $\mathcal{H}_2$ control \cite[Def 15]{waarde2020noisy}}{
	the data $(U_-,X)$ is informative for $\mathcal{H}_2$ control with performance $\gamma$ if there exists a symmetric positive definite matrices $p$ and a matrix $K$ such that:
	\begin{align*} \tag{\ref{H2Inequalities}}
		P &> (A + BK)^\top P (A + BK) + (C + DK)^\top(C+DK) & 
		trace(P) &< \gamma^2
	\end{align*}
	holds for all $(A,B) \in \Sigma_{i/s/n}$.
}

% Mathematics
At this point, all that is left is to rewrite the inequality (\ref{H2Inequalities}) such that we are able to apply the previously discussed matrix S-lemma. Using similar techniques as in quadratic stabilisation we can derive the following theorem. For more information about this, we will refer the reader to \cite[Section V.A]{waarde2020noisy}.

%To do this, we will first introduce some notation. Let $Y = P^{01}$ and $L = KY$. Then we are able to write (\ref{H2Inequalities}) as:
%\begin{equation*}
%	Y - (AY + BL)^\top P (AY + BL) - (CY + DL)^\top (CY+DL) > 0
%\end{equation*}

%\todo{Using a schur complement argument, this is equivalent to}

%\begin{equation} \label{H2AfterSchur}
%	\begin{bmatrix} Y - (CY + DL)^\top(CY + DL) & (AY + BL)^\top \\ AY + BL & Y \end{bmatrix} > 0
%\end{equation}

%Note that this only holds if the following 2 condition hold.
%\begin{align*}
%	Y - (CY + DL)^\top(CY + DL) &> 0 \\
%	Y - (AY + BL) \left(Y - (CY + DL)^\top(CY + DL)\right)^{-1} (AY + BL)^\top &> 0
%\end{align*}

%Since the first equation is independent of $A$ and $B$, we are able to write it as follows.

%\begin{equation*}
%	\left[\begin{array}{c}
%		I \\ \hline A^\top \\ B^\top
%	\end{array}\right]^\top
%	\left[\begin{array}{c|c}
%		Y & 0 \\ \hline
%		0 & \begin{bmatrix} Y \\ L \end{bmatrix} \left(Y - (CY + DL)^\top(CY + DL)\right)^{-1} \begin{bmatrix} Y \\ L \end{bmatrix}^\top
%	\end{array}\right]
%	\left[\begin{array}{c}
%		I \\ \hline A^\top \\ B^\top
%	\end{array}\right] > 0
%\end{equation*}

%As we can see the inequality is in a similar form as (\ref{noiseSystemQMI}) and thus via the S-lemma to get the following theorem.

% State theorem
\Thr{\cite[Thr 16]{waarde2020noisy}}{
	Assume that the generalised Slater condition (\ref{GenerelisedSlaterCondition}) holds for $N$ in (\ref{NData}) and some $\bar{Z} \in \mathbb{R}^{(n+m_ \times n)}$. Then the data $(U_-,X)$ is informative for $\mathcal{H}_2$ control with performance $\gamma$ if and only if there exists matrices $Y = Y^\top > 0$, $Z = Z^\top$ and $L$, and scalars $\alpha \geq 0$ and $\beta>0$ satisfying:
	\begin{equation*}
		\begin{bmatrix}
			Y - \beta I & 0 & 0 & 0 & 0 \\
			0 & 0 & 0 & Y & 0 \\
			0 & 0 & 0 & L & 0 \\
			0 & Y & L^\top & Y & (CY + DL)^\top \\
			0 & 0 & 0 & CY + DL & I 
		\end{bmatrix} - \alpha
		\begin{bmatrix} I&X_+ \\ 0 & -X_- \\ 0&-U_- \\ 0&0 \\ 0&0 \end{bmatrix}
		\begin{bmatrix} \Phi_{11} & \Phi_{12} \\ \Phi_{12}^\top & \Phi_{22} \end{bmatrix}
		\begin{bmatrix} I&X_+ \\ 0 & -X_- \\ 0&-U_- \\ 0&0 \\ 0&0 \end{bmatrix}^\top \geq 0
	\end{equation*}
	\begin{align*}
		\begin{bmatrix} Y & (CY + DL)^\top \\ CY + DL & I \end{bmatrix} &> 0 &
		\begin{bmatrix} Z & I \\ I & Y \end{bmatrix} &\geq 0 &
		trace(Z) &< \gamma^2
 	\end{align*}
	Moreover, if $Y$ and $L$ satisfy the above conditions, then $K = LY^{-1}$ is such that $A + BK$ is stable and $\| G(z)\|_{\mathcal{H}_2} < \gamma$ for all $(A,B) \in \Sigma$.
}

This theorem was implemented in the functions below.

% Pseudo code / algorithm
\subsection{Implementation}
\subsubsection*{Syntax} 
\mon{[bool, K, diagnostics, gamma, info] = isInformH2(X, U, Phi, C, D)} \\
\mon{[bool, K, diagnostics, gamma, info] = isInformH2(X, U, Phi, C, D, tolerance)} \\
\mon{[bool, K, diagnostics, gamma, info] = isInformH2(X, U, Phi, C, D, tolerance, options)} 

\subsubsection*{Description}
\mon{[bool, K, diagnostics, gamma, info] = isInformH2(X, U, Phi, C, D)}: Checks is the data is informative for \mbox{H2} control. \\
\mon{[bool, K, diagnostics, gamma, info] = isInformH2(X, U, Phi, C, D, tolerance)}: Checks is the data is informative for \mbox{H2} control using a given tolerance. \\
\mon{[bool, K, diagnostics, gamma, info] = isInformH2(X, U, Phi, C, D, tolerance, options)}: Checks is the data is informative for \mbox{H2} control using a given tolerance and a given spdsetting object from Yalmip.

\subsubsection*{Input arguments}
\textbf{\mon{X}}: State data matrix of dimension $n \times T+1$ from a input/state data set.\\
\textbf{\mon{U}}: Input data matrix of dimension $m \times T$ from a input/state data set.\\
\textbf{\mon{Phi}}: Noise matrix as in (\ref{noiseBound}). \\ 
\textbf{\mon{C}}: State performance matrix. \\ 
\textbf{\mon{D}}: State performance matrix. \\ 
\textbf{\mon{tolerance}}: Tolerance used for determining when a value is zero up to machine precision. Default value is \mon{1e-8}.\\
\textbf{\mon{options}}: sdpsettings used with the Yalmip solver.

\subsubsection*{Output arguments}
\textbf{\mon{bool}}: (boolean) True if the data is informative for quadratic stabilisation, false otherwise. If false then the \mon{info} variable can be check to find which condition failed. \\
\textbf{\mon{K}}: (matrix) If the data is informative, it contains a stabilising controller \mon{K} for closed loop control \mon{A+BK}, empty otherwise.\\
\textbf{\mon{diagnostics}}: (struct) Diagnostics from the Yalmip \mon{optimize()} function. \\
\textbf{\mon{gamma}}: (double) Value of gamma as found by the solver. \\
\textbf{\mon{info}}: (int) Diagnostic variable use to identify which conditions (if any) failed. The verification is done on the solution obtained from Yalmip. For information about the type of error use the \mon{help} command in Matlab

\subsubsection*{Limitation}
Due to computational limitation in the solver or conditioning of this problem it might be the case that a valid \mon{K} is found even though \mon{bool} is false. This is because we are able to find a solution close enough to the real solution that the results are still satisfactory, even though not all conditions are met. 

% Example using function
\subsection{Example} \label{ExampleH2}
In this example we will consider the same data that we used in the quadratic stabilisation example. However, we will be adding a performance output $z(t)$ to the system.
\begin{align*}
	x(t+1) &= x(t) - \frac{1}{2}u(t) + w(t) \\
	z(t)   &= \frac{1}{2}x(t) \\
	W_- &= \begin{bmatrix}  0.0706 &  0.0244 & -0.0298 & 0.0026 & -0.0196 \end{bmatrix} \\
	U_- &= \begin{bmatrix} -0.1655 & -0.9007 &  0.8054 & 0.8896 & -0.0183 \end{bmatrix} \\
	X   &= \begin{bmatrix}     0   &  0.1533 &  0.6281 & 0.1956 & -0.2466 & -0.2571 \end{bmatrix} \\
\end{align*}

As we saw in that example we can use the following $\Phi$ as a valid noise bound.

\begin{equation*}
\Phi = \begin{bmatrix} 0.05 & \overline{0} \\ \overline{0}^\top & -I_5 \end{bmatrix}
\end{equation*}

We also know that the generalised Slater condition holds. Note that we picked $C = 0.5$ and $D=0$ Because of this we penalise the state being non zero. Hence the optimal controller will be a deadbeat controller. 

At this point we only need to find a $Y = Y^\top > 0$, $Z = Z^\top$, $L$, $\alpha$ and $\beta$ such that the following inequalities hold.

\begin{equation*}
	\begin{bmatrix}
		Y - \beta I & 0 & 0 & 0 & 0 \\
		0 & 0 & 0 & Y & 0 \\
		0 & 0 & 0 & L & 0 \\
		0 & Y & L^\top & Y & (CY + DL)^\top \\
		0 & 0 & 0 & CY + DL & I 
	\end{bmatrix} - \alpha
	\begin{bmatrix} I&X_+ \\ 0 & -X_- \\ 0&-U_- \\ 0&0 \\ 0&0 \end{bmatrix}
	\begin{bmatrix} \Phi_{11} & \Phi_{12} \\ \Phi_{12}^\top & \Phi_{22} \end{bmatrix}
	\begin{bmatrix} I&X_+ \\ 0 & -X_- \\ 0&-U_- \\ 0&0 \\ 0&0 \end{bmatrix}^\top \geq 0
\end{equation*}

\begin{align*}
	\begin{bmatrix} Y & (CY + DL)^\top \\ CY + DL & I \end{bmatrix} &> 0 &
	\begin{bmatrix} Z & I \\ I & Y \end{bmatrix} &\geq 0 &
	trace(Z) &< \gamma^2
\end{align*}

Using numerical tools like Yalmip we are able to find that the following values form a valid solution.

\begin{align*}
	Y &= 3.578190407584625 \\
	Z &= 0.279470874690682 \\
	L &= 6.860884484332921 \\
	a &= 80.398493367163880 \\
	b &= 1.162504055224782e-06
\end{align*}
With a performance of $\gamma = 0.528650049362224$ and a feedback gain $K = 1.917417382202476$. This will result in a closed loop system which is 'close' to deadbeat stable.
\begin{equation*}
	A + BK = 0.0413
\end{equation*}

We are also able to find a stabilising feedback gain using the provided Matlab function.
\begin{lstlisting}
C = 0.5;
D = 0;
U = [-0.1655 -0.9007 0.8054 0.8896 -0.0183];
X = [ 0       0.1533 0.6281 0.1956 -0.2466 -0.2571];
Phi = [0.05 zeros(1,5) ; zeros(5,1) -eye(5)];
[bool, K] = isInformH2(X, U, Phi, C, D)
\end{lstlisting}
Which will return: \mon{[ 1, 1.917417382202475 ]}.

Since we have the 'true' system, we are also able to find the best possible $\gamma_{best}$ and $K_{best}$. For this we will use on of the intemediary steps of the theorem \cite[Equation 36]{waarde2020noisy}. We will use the following code to compute the optimal gain.
\begin{lstlisting}
A = 1;
B = -0.5;
C = 0.5;
D = 0;
L = sdpvar(1);
Y = sdpvar(1);

Cond = [ [Y - (C*Y+D*L)' * (C*Y+D*L) (A*Y+B*L)' ; (A*Y+B*L) Y] >= 0 ];
Cond = Cond + [Y >= 1e-8];
optimize(Cond,-Y);

gamma_best = sqrt(trace(inv(value(Y))))
K_best = value(L) * inv(value(Y))
\end{lstlisting}
Which will return: \mon{gamma\_best = 0.5} and \mon{K\_best = 2};

From this we can see that the optimal feedback gain is indeed a deadbeat controller. We can verify that calculated gamma using (\ref{H2Inequalities}). Since we are considering the 1 dimensional case, we have that $\gamma^2 > P = trace(P)$, thus we can rewrite the equation to get the following. 
\begin{align*}
	\gamma^2 &> (A + BK)^2 P + (C+DK)^2 \\
	\gamma^2 &>  C^2
\end{align*}
Note that since $P>0$ we want that $(A+BK) = 0$ to have the best gamma posible. As we can see, for our values of $(A,B,C,D)$ the optimal solution is $K_{best} = 2$ with $\gamma_{best} = \frac{1}{2}$.