\section{Dynamic measurement feedback}
% Abstract section


% What is DMF


% State identification


% Mathematics


% Proof?


% Pseudo code / algorithm


% Example using function
\subsection{Examples using implementation}
The algorithm above is implemented in the following functions:
\subsubsection*{Syntax} 
\mon{[bool, K, L, M] = isInformDynamicMeasurementFeedback(X, U, Y)} \\
\mon{[bool, K, L, M] = isInformDynamicMeasurementFeedback(X, U, Y, polesM)} \\
\mon{[bool, K, L, M] = isInformDynamicMeasurementFeedback(X, U, Y, polesM, polesL)}

\subsubsection*{Description} 
\mon{[bool, K, L, M] = isInformDynamicMeasurementFeedback(X, U, Y)}: \\
\mon{[bool, K, L, M] = isInformDynamicMeasurementFeedback(X, U, Y, polesM)}: \\
\mon{[bool, K, L, M] = isInformDynamicMeasurementFeedback(X, U, Y, polesM, polesL)}: 

\subsubsection*{Input arguments} 
\textbf{\mon{X}}: State data matrix of dimension $n \times T+1$.\\
\textbf{\mon{U}}: Input data matrix of dimension $m \times T$.\\
\textbf{\mon{Y}}: Output data matrix of dimension $p \times T$.\\
\textbf{\mon{polesM}}: .\\
\textbf{\mon{polesL}}: .

\subsubsection*{Output arguments} 
\textbf{\mon{bool}}: (boolean) True if the data is informative for dynamic measurement feedback, false otherwise\\
\textbf{\mon{K}}: (matrix) .\\
\textbf{\mon{L}}: (matrix) .\\
\textbf{\mon{M}}: (matrix) .


\subsubsection*{Syntax}
\mon{[bool, X\_bar, U\_bar, Y\_bar] = isInformStateIdentification(U, Y, n)}

\subsubsection*{Description}
\mon{[bool, X\_bar, U\_bar, Y\_bar] = isInformStateIdentification(U, Y, n)}: 

\subsubsection*{Input arguments}
\textbf{\mon{U}}: Input data matrix of dimension $m \times T$.\\
\textbf{\mon{Y}}: Output data matrix of dimension $p \times T$.\\
\textbf{\mon{n}}: Dimension of the state space.

\subsubsection*{Output arguments}
\textbf{\mon{bool}}: (boolean) True if the data is informative for state space identification, false otherwise\\
\textbf{\mon{X\_bar}}: (matrix) .\\
\textbf{\mon{U\_bar}}: (matrix) .\\
\textbf{\mon{Y\_bar}}: (matrix) .

\subsubsection{Examples}