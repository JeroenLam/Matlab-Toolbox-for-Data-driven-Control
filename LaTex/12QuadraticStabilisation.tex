\section{Quadratic stabilisation} \label{SectionQuadStab}
% Abstract section
In this section we will start by formulating the quadratic stabilisation problem for data driven control. We will rewrite the problem to a quadratic matrix inequality ans solve it using the matrix S-lemma. Lastly we will see the implementation of the described methods and an example on its application.

\subsection{Problem formulation}
% Problem definition of QS
We will start by defining when data is informative for quadratic stabilisation. We will go into more detail about this type of control in the later section \ref{SectionQuadStab}. For now we will use this definition to give an intuitive idea on how we end up at the final theory.

\Def{Informative for quadratic stabilisation \cite[Def 3]{waarde2020noisy}}{
	The data $(U_-,X)$ is called informative for quadratic stabilisation if there exists a feedback gain $K$ and a Lyapunov matrix $P = P^\top > 0$ such that $P - (A + BK) P (A + BK)^\top > 0$ for all $(A,B) \in \Sigma_{i/s/n}$
}

% Note that this is equivalent to the matrix inequalities
We can expend the matrix inequality to get the following.
\begin{equation} \label{noiseQSQMI}
P - (A + BK) P (A + BK)^\top = 
\begin{bmatrix} I\\A^\top\\B^\top \end{bmatrix}^\top 
\begin{bmatrix} P&0&0\\0&-P&-PK^\top\\0&-KP&-KPK^\top \end{bmatrix} 
\begin{bmatrix} I\\A^\top\\B^\top \end{bmatrix} > 0
\end{equation}
For the sake of argument, lets pick $P$ and $K$ to be fixed. Then our control problem reduces to knowing when there is sufficient overlap in both solution sets of the quadratic matrix inequalities (\ref{noiseQSQMI}) and (\ref{noiseSystemQMI}), or in other words, when does one quadratic matrix inequality imply another quadratic matrix inequality.


% Matrix S-Lemma and Generelised Slater condition
To see when one quadratic matrix inequality implies another matrix inequality, we can use the a theorem based on the strict matrix S-lemma as presented in \cite{waarde2020noisy}. However before we look at the theorem, we will first consider the generalised Slater condition.

\Thr{\cite[Thr 12]{waarde2020noisy}}{
	Let $M, N \in \mathbb{R}^{(k+n) \times (k+n)}$ be symmetric matrices, partitioned as:
	\begin{align*}
		M &= \begin{bmatrix} M_{11} & M_{12} \\ M_{12}^\top & M_{22} \end{bmatrix} & N &= \begin{bmatrix} N_{11} & N_{12} \\ N_{12}^\top & N_{22} \end{bmatrix}
	\end{align*}
	Assume that $M_{22} \leq 0$, $N_{22} \leq 0$ and $ker \, N_{22} \subseteq ker \, N_{12}$. Suppose that there exists some matrix $\bar{Z} \in \mathbb{R}^{n\times k}$ satisfying the generalised Slater condition:
	\begin{equation} \label{GenerelisedSlaterCondition}
		\begin{bmatrix} I\\\bar{Z} \end{bmatrix}^\top N 	\begin{bmatrix} I\\\bar{Z} \end{bmatrix} > 0. 
	\end{equation}
	Then we have that
	\begin{equation} \label{QStheoremCondition}
		\begin{bmatrix} I \\ Z \end{bmatrix}^\top M \begin{bmatrix} I \\ Z \end{bmatrix} > 0 \mbox{ for all } Z \in \left\{ Z \in \mathbb{R}^{n\times k} \, : \, \begin{bmatrix} I \\ Z \end{bmatrix}^\top N \begin{bmatrix} I \\ Z \end{bmatrix} \geq 0 \right\} 
	\end{equation}
	if and only if there exist $\alpha \geq 0$ and $\beta > 0$ such that 
	\begin{equation*}
		M - \alpha N \geq \begin{bmatrix} \beta I & 0 \\ 0 & 0 \end{bmatrix}
	\end{equation*}
}

%\Thr{Matrix S-lemma \cite[Theorem 9]{waarde2020noisy}}{
%	Let $M$, $N \in \mathbb{R}^{(n+k) \times (n+k)}$ be symmetric matrices and assume that there exists some matrix $\bar{Z} \in mathbb{R}^{n\times k}$ such that
%	\begin{equation} \label{GenerelisedSlaterCondition}
%	\begin{bmatrix} I\\\bar{Z} \end{bmatrix}^\top N 	\begin{bmatrix} I\\\bar{Z} \end{bmatrix} > 0. 
%	\end{equation}
%	Then the following statements are equivalent:
%	\begin{enumerate}
%		\item $\begin{bmatrix} I\\Z \end{bmatrix}^\top M \begin{bmatrix} I\\Z \end{bmatrix} \geq 0$ for all $Z\in\mathbb{R}^{n \times k}$ with $\begin{bmatrix} I\\Z \end{bmatrix}^\top N \begin{bmatrix} I\\Z \end{bmatrix} \geq 0$.
%		\item $\begin{bmatrix} I\\Z \end{bmatrix}^\top M \begin{bmatrix} I\\Z \end{bmatrix} \geq 0$ for all $Z\in\mathbb{R}^{n \times k}$ with $\begin{bmatrix} I\\Z \end{bmatrix}^\top N \begin{bmatrix} I\\Z \end{bmatrix} > 0$.
%		\item There exists a scalar $\alpha \geq 0$ such that $M - \alpha N \geq 0$.
%	\end{enumerate}
%}
%Similar to the classical S-lemma, the assumption (\ref{GenerelisedSlaterCondition}) is called the generalised Slater condition.

Using this theorem we are able to get find a solution to our quadratic matrix inequality problem. But before we look at the solution, we will first consider how we can verify that the generalised Slater condition holds for a given matrix $N$. For this we will consider the following theorem which gives us an easy to compute condition for verifying the Slater condition.

\Thr{}{
	Let $N \in \mathbb{R}^{(k+n)\times (k+n)}$ be symmetric. Then the following are equivalent.
	\begin{enumerate}
		\item There exists a $Z \in \mathbb{R}^{n\times k}$ such that $\begin{bmatrix} I_k\\\bar{Z} \end{bmatrix}^\top N 	\begin{bmatrix} I_k\\\bar{Z} \end{bmatrix} > 0$.
		\item $N$ has at least $k$ positive eigenvalues.
	\end{enumerate}
}

%Proof '$1 \Rightarrow 2$': \\
%\todo{fdsa}

%Proof '$1 \Leftarrow 2$': \\
%\todo{fdsa}

This theorem is implemented in the following function.

% Pseudo code / algorithm
\subsection{Implementation Slater condition}
\subsubsection*{Syntax}
\mon{[bool] = testSlater(X, U, Phi)} 

\subsubsection*{Description}
\mon{[bool] = testSlater(X, U, Phi)}: Returns if the Slater condition holds for a matrix \mon{N} constructed from the data \mon{(X,U)} and the noise matrix \mon{Phi}.

\subsubsection*{Input arguments}
\textbf{\mon{X}}: State data matrix of dimension $n \times T+1$ from a input/state data set.\\
\textbf{\mon{U}}: Input data matrix of dimension $m \times T$ from a input/state data set.\\
\textbf{\mon{Phi}}: Noise matrix as in (\ref{noiseBound}).

\subsubsection*{Output arguments}
\textbf{\mon{bool}}: (boolean) True if the Slater condition holds, false otherwise.


\subsection{Solving the data driven quadratic stabilisation problem.}
We will recall that the quadratic stabilisation matrix inequality can be rewritten as:
\begin{equation} \tag{\ref{noiseQSQMI}}
P - (A + BK) P (A + BK)^\top = 
\begin{bmatrix} I\\A^\top\\B^\top \end{bmatrix}^\top 
\begin{bmatrix} P&0&0\\0&-P&-PK^\top\\0&-KP&-KPK^\top \end{bmatrix} 
\begin{bmatrix} I\\A^\top\\B^\top \end{bmatrix} > 0
\end{equation}

As alluded to earlier, we want to find a condition in which case (\ref{noiseQSQMI}) holds as well as:
\begin{equation} \tag{\ref{noiseSystemQMI}}
\begin{bmatrix} I \\ A^\top \\ B^\top \end{bmatrix}^\top
\begin{bmatrix} I&X_+ \\ 0 & -X_- \\ 0&-U_- \end{bmatrix}
\begin{bmatrix} \Phi_{11} & \Phi_{12} \\ \Phi_{12}^\top & \Phi_{22} \end{bmatrix}
\begin{bmatrix} I&X_+ \\ 0 & -X_- \\ 0&-U_- \end{bmatrix}^\top
\begin{bmatrix} I \\ A^\top \\ B^\top \end{bmatrix} \geq 0
\end{equation}

% Combine with S-lemma to construct (fs)
Before we apply the theorem we first need to partition our matrices and check the preliminary conditions. We will start by partitioning our matrices in the following way.
\begin{subequations} \label{NData}
	\begin{align} 
		M &= 
		\left[ \begin{array}{c|c}
			M_{11} & M_{12} \\ \hline M_{12}^\top & M_{22}
		\end{array} \right] 
		:=
		\left[ \begin{array}{c|cc}
		P&0&0 \\ \hline 0 & -P & -PK^\top \\ 0 & -KP & -KPK^\top
		\end{array} \right] \\
		N &= 
		\left[ \begin{array}{c|c}
			N_{11} & N_{12} \\ \hline N_{12}^\top & N_{22}
		\end{array} \right] 
		:=
		\left[ \begin{array}{cc}
			I&X_+ \\ \hline 0 & -X_- \\ 0&-U_-
		\end{array} \right]
		\left[ \begin{array}{cc}
			\Phi_{11} & \Phi_{12} \\ \Phi_{12}^\top & \Phi_{22}
		\end{array} \right]
		\left[ \begin{array}{cc}
			I&X_+ \\ \hline 0 & -X_- \\ 0&-U_-
		\end{array} \right]^\top
	\end{align}
\end{subequations}

Now we need to check if $N_{22} \leq 0$ and $M_{22} \leq 0$. 
\begin{align*}
	N_{22} &= \begin{bmatrix} X_- \\ U_- \end{bmatrix} \Phi_{22} \begin{bmatrix} X_- \\ U_- \end{bmatrix}^\top \leq 0 &
	M_{22} &= - \begin{bmatrix} I \\ K \end{bmatrix} P \begin{bmatrix} I \\ K \end{bmatrix}^\top \leq 0
\end{align*}
Since $P$ is symmetric positive definite and $\Phi_{22}$ is symmetric negative definite. 

Now we only need to check that $ker \, N_{22} \subseteq ker \, N_{12}$. Since $\Phi_{22}$ is symmetric negative definite we know that:
\[ ker \, N_{22} = ker \begin{bmatrix} X_- \\ U_- \end{bmatrix}^\top \]
If we write out $N_{12}$ we get the following:
\begin{align*}
	 N_{12} &= (\Phi_{12} + X_+ \Phi_{22}) \begin{bmatrix} -X_- \\ -U_- \end{bmatrix}^top \\
	 ker \, N_{12} &= ker -(\Phi_{12} + X_+ \Phi_{22}) \begin{bmatrix} X_- \\ U_- \end{bmatrix}^top
\end{align*}
Thus we can conclude that $ker \, N_{22} \subseteq ker \, N_{12}$. 

Hence we assume the generalised Slater condition holds, we know that (\ref{QStheoremCondition}) holds if and only if there exist $\alpha \geq 0$ and $\beta > 0$ such that:
\begin{equation*}
	M - \alpha N \geq \begin{bmatrix} \beta I & 0 \\ 0 & 0 \end{bmatrix}
\end{equation*}

Using this we can construct the following theorem.

% State theorem
\Thr{\cite[Thr 13]{waarde2020noisy}}{
	Assume that the generalised Slater condition (\ref{GenerelisedSlaterCondition}) holds for $N$ in (\ref{NData}) and some $\bar{Z} \in \mathbb{R}^{(n+m_ \times n)}$. Then the data $(U_-,X)$ is informative for quadratic stabilisation if and only if there exists an $n \times n$ matrix $P = P^\top > 0$, an $L \in \mathbb{R}^{m\times n}$ and scalars $\alpha \geq 0$ and $\beta > 0$ satisfying:
	\begin{equation} \label{QuadStabCondition}
	\begin{bmatrix}
		P-\beta I&0&0&0\\0&-P&-L^\top&0\\0&-L&0&L\\0&0&L^\top&P
	\end{bmatrix} - \alpha 
	\begin{bmatrix} I&X_+ \\ 0 & -X_- \\ 0&-U_-\\0&0 \end{bmatrix}
	\begin{bmatrix} \Phi_{11} & \Phi_{12} \\ \Phi_{12}^\top & \Phi_{22} \end{bmatrix}
	\begin{bmatrix} I&X_+ \\ 0 & -X_- \\ 0&-U_-\\0&0 \end{bmatrix}^\top \geq 0
	\end{equation}
	Moreover, if $P$ and $L$ satisfy this matrix inequality, then $K = LP^{-1}$ is a stabilising feedback gain for all $(A,B) \in \Sigma$.
}

Note that a change of coordinates is applied since the original $M$ matrix was not linear in $P$ and $K$. This theorem is incorporated in the following theorem.

% Pseudo code / algorithm
\subsection{Implementation}
\subsubsection*{Syntax}
\mon{[bool, K, diagnostics, info] = isInformQuadraticStabilisation(X, U, Phi)} \\
\mon{[bool, K, diagnostics, info] = isInformQuadraticStabilisation(X, U, Phi, tolerance)} \\
\mon{[bool, K, diagnostics, info] = isInformQuadraticStabilisation(X, U, Phi, tolerance, options)} 

\subsubsection*{Description}
\mon{[bool, K, diagnostics, info] = isInformQuadraticStabilisation(X, U, Phi)}: Checks if the data is informative for quadratic stabilisation. \\
\mon{[bool, K, diagnostics, info] = isInformQuadraticStabilisation(X, U, Phi, tolerance)}: Checks if the data is informative for quadratic stabilisation using a given tolerance. \\
\mon{[bool, K, diagnostics, info] = isInformQuadraticStabilisation(X, U, Phi, tolerance, options)}: Checks if the data is informative for quadratic stabilisation using a given tolerance and a given spdsetting object from Yalmip.

\subsubsection*{Input arguments}
\textbf{\mon{X}}: State data matrix of dimension $n \times T+1$ from a input/state data set.\\
\textbf{\mon{U}}: Input data matrix of dimension $m \times T$ from a input/state data set.\\
\textbf{\mon{Phi}}: Noise matrix as in (\ref{noiseBound}). \\ 
\textbf{\mon{tolerance}}: Tolerance used for determining when a value is zero up to machine precision. Default value is \mon{1e-12}.\\
\textbf{\mon{options}}: sdpsettings used with the Yalmip solver.

\subsubsection*{Output arguments}
\textbf{\mon{bool}}: (boolean) True if the data is informative for quadratic stabilisation, false otherwise. If false then the \mon{info} variable can be check to find which condition failed. \\
\textbf{\mon{K}}: (matrix) If the data is informative, it contains a stabilising controller \mon{K} for closed loop control \mon{A+BK}, empty otherwise.\\
\textbf{\mon{diagnostics}}: (struct) Diagnostics from the Yalmip \mon{optimize()} function. \\
\textbf{\mon{info}}: (int) Diagnostic variable use to identify which conditions (if any) failed. The verification is done on the solution obtained from Yalmip. For information about the type of error use the \mon{help} command in Matlab

\subsubsection*{Limitation}
Due to computational limitation in the solver or conditioning of this problem it might be the case that a valid \mon{K} is found even though \mon{bool} is false. This is because we are able to find a solution close enough to the real solution that the results are still satisfactory, even though not all conditions are met. 

% Example using function
\subsection{Example} \label{ExampleQS}
In this example we will consider the following unstable system.
\begin{align*}
	x(t+1) &= x(t) - \frac{1}{2}u(t) + w(t) 
\end{align*}
We will assume our noise to be picked uniformly from the interval $[-0.1 , 0.1]$. We will start by generating a random noise sequence and a random input sequence which we will use to construct the following data.
\begin{align*}
	W_- &= \begin{bmatrix}  0.0706 &  0.0244 & -0.0298 & 0.0026 & -0.0196 \end{bmatrix} \\
	U_- &= \begin{bmatrix} -0.1655 & -0.9007 &  0.8054 & 0.8896 & -0.0183 \end{bmatrix} \\
	X   &= \begin{bmatrix}     0   &  0.1533 &  0.6281 & 0.1956 & -0.2466 & -0.2571 \end{bmatrix} \\
\end{align*}

For our noise matrix $\Phi$ we will use that if we pick $\Phi_{12} = \overline{0} = [0 \dots 0]$ and $\Phi_{22} = -I$ then we know that $W_- W_-^\top \leq \Phi_{11}$. Since we know that our data is sampled from a uniform distribution with a maximum value of $0.1$ we know that $W_- W_-^\top \leq 0.1^2 T$ where $T$ is the total number of samples considered. Since we are considering $5$ sample we will pick our noise matrix to be:
\begin{equation*}
	\Phi = \begin{bmatrix} 0.05 & \overline{0} \\ \overline{0}^\top & -I_5 \end{bmatrix}
\end{equation*}
Using some computation we can verify that this choice of $\Phi$ is indeed a valid one for the data.
\begin{equation}
\begin{bmatrix} I \\ W_-^\top \end{bmatrix} ^\top
\begin{bmatrix} \Phi_{11} & \Phi_{12} \\ \Phi_{12}^\top & \Phi_{22} \end{bmatrix}
\begin{bmatrix} I \\ W_-^\top \end{bmatrix} = 0.0431 \geq 0
\end{equation}

Before we attempt to solve the matrix inequality, we need to check if the generalised Slater condition holds. As we have seen in the previous section, we only have to verify that $N$ has at least 1 positive eigenvalue.
\begin{equation*}
 N = 
 \begin{bmatrix} I&X_+ \\ 0 & -X_- \\ 0&-U_- \end{bmatrix}
 \begin{bmatrix} 0.05 & \overline{0} \\ \overline{0}^\top & -I_5 \end{bmatrix}
 \begin{bmatrix} I&X_+ \\ 0 & -X_- \\ 0&-U_- \end{bmatrix}^\top 
 =
 \begin{bmatrix} 
 -0.5332 &  0.2343 & -0.6482\\
  0.2343 & -0.5171 & -0.5463 \\
 -0.6482 & -0.5463 & -2.2790
 \end{bmatrix} 
\end{equation*}
\begin{equation*}
 \sigma(N) = \{ \begin{array}{ccc}
 -2.5921 & -0.7571 & 0.02
 \end{array} \}
\end{equation*}

Now that we have our data prepared, we will attempt to find a $P > 0$, $L$, $\alpha \geq 0$ and $\beta > 0$ such that (\ref{QuadStabCondition}) holds.
\begin{align*}
	\begin{bmatrix}
		P-\beta I&0&0&0\\0&-P&-L^\top&0\\0&-L&0&L\\0&0&L^\top&P
	\end{bmatrix} &- \alpha 
	\begin{bmatrix} I&X_+ \\ 0 & -X_- \\ 0&-U_-\\0&\overline{0} \end{bmatrix}
	\begin{bmatrix} 0.05 & \overline{0} \\ \overline{0}^\top & -I_5 \end{bmatrix}
	\begin{bmatrix} I&X_+ \\ 0 & -X_- \\ 0&-U_-\\0&\overline{0} \end{bmatrix}^\top \geq 0 
\end{align*}
Using numerical tools such as Yalmip we can find that the following values form a valid solution to the matrix inequality.
\begin{align*}
	P &= \frac{3}{10} & L &= \frac{1}{2} & \alpha &= \frac{9}{10} & \beta &= \frac{1}{10}
\end{align*}
Using these values we are able to construct a stabilising controller $K = LP^{-1} = 1\frac{2}{3}$. As we can see the closed loop system is indeed stable.
\begin{equation*}
	A + BK = \frac{1}{6}
\end{equation*}

We are also able to find a stabilising feedback gain using the provided Matlab function.
\begin{lstlisting}
U = [-0.1655 -0.9007 0.8054 0.8896 -0.0183];
X = [ 0       0.1533 0.6281 0.1956 -0.2466 -0.2571];
Phi = [0.05 zeros(1,5) ; zeros(5,1) -eye(5)];
[bool, K] = isInformQuadraticStabilisation(X, U, Phi)
\end{lstlisting}
Which will return: \mon{[ 1, 1.6977 ]}.
























